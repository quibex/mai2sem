\documentclass[a4paper,12pt,cmcyralt]{article}

\usepackage[english, russian]{babel}
\usepackage[utf8]{inputenc}
\usepackage{amsmath}
\usepackage{amsfonts}

\begin{document}
\setcounter{page}{263}
При использовании равенств с символами $ O $ и $ o $ следует иметь в виду, что они не являются равенствами в обычном смысле этого слова. Так, если $\alpha_1 = o(\beta)$ при $x \to x_0$, $\alpha_2 = o(\beta)$ при $x \rightarrow x_0$, то было бы ошибкой сделать отсюда заключение, что $\alpha_1 = \alpha_2$, как это было бы в случае обычных равенств. Например, $ x^3 = o(x) $ и $ x^2 = o(x) $  при $ x\to0$, но $x^2 \ne x^3$.

Аналогично, если $f + O(f) = g + O(f) $ при $x\rightarrow x_0$, то было бы ошибкой сделать заключение, что $f = g$.
 
Дело в том, что один и тот же символ $O(f)$ или $o(f)$ может обозначать разные конкретные функции. Это обстоятельство связано с тем, что при определении символов $O(f)$ и $o(f)$ мы по существу ввели целые классы функций, обладающих определенными свойствами (класс функций, ограниченных в некоторой окрестности точки $х_0$ по сравнению с функцией $f$, и класс функций, бесконечно малых посравнению с $f(x)$ при $x\rightarrow x_0$), и было бы правильнее писать не $\alpha = O(f)$ и $\alpha = o(f)$, a соответственно $\alpha \in O(f)$ на и $\alpha \in o(f)$. Однако это привело бы к существенному усложнению вычислений по формулам, в которых встречаются символы $O$ и $o$. Поэтому мы сохраним прежнюю запись $\alpha = O(f)$ и $\alpha = o(f)$, но будем всегда читать эти равенства, в соответствии с приведенными выше определениями, только в одну сторону; слева направо (если, конечно, не оговорено что-либо другое). Например, запись:
\[ \alpha = o(f),x\rightarrow x_0 \]
означает, что функция $\alpha$ является бесконечно малой по сравнению с функцией $f$ при $x\rightarrow x_0$, но отнюдь не то, что всякая бесконечно малая по сравнению с функция равна с.

В качестве примера использования этих символов докажем равенство где с постоянная.
$$ o(cf) = o(f),\eqno (8.34) $$ где $c$ --- постоянная.

Согласно сказанному, надо показать, что если g $= o(cf)$, то  g $= o(f)$. Действительно, если g $= о(cf)$, то g = $\epsilonсf$, где $\lim\limits_{x\to x_0} \epsilon(x)=0$. Положим $\epsilon_1 = c\epsilon $; тогда  g $= \epsilon_1 f$, где, очевидно, $\lim\limits_{x\to x_0} \epsilon_1(x)=0$ и, значит, g $= o(f)$.

В заключение отметим, что сказанное об использовании символов $o$ и $O$ не исключает, конечно, того, что отдельные формулы с этими символами могут оказаться справедливыми не только при чтении слева направо, но и справа налево: так, формула (8.34) при $ c \ne 0$ верна и при чтении справа налево.

\setcounter{section}{8}
\setcounter{subsection}{2}
\noindent\subsection{\textsf{Эквивалентные функции}}


\noindentЕсли функция $f(x)$ заменяется функцией $g(x)$, то разность $f(x) - g(x)$ называется \emабсолютной погрешностью\em, а отношение $\frac{f(x) - g(x)}{f(x)}$ -- \emотносительной погрешностью \em сделанной замены. Если изучается поведение функции $f(x)$ при $x \rightarrow x_0$, то часто целесообразно заменить ее функцией $g(x)$ такой, что: 1) функция $g(x)$ в определенном смысле более простая, чем функция $f(x)$; 2) абсолютная погрешность стремится к нулю при $x \rightarrow x_0$:
$$ \lim\limits_{x\to x_0} [f(x) - g(x)] =0 .$$

В этом случае говорят, что $g(x)$ приближает или аппрок симирует функцию $f(x)$ вблизи точки $x_0$. Таким свойством обладают, например, все бесконечно малые при $x\to x_0$ функции $f$ и $g$.

Ниже будет показано, что среди них лишь те, которые эквивалентны между собой: $g(x)~f(x)$, $x\to x_0$, обладают тем свойством, что не только абсолютная погрешность $f(x) - g(x)$, но и относительная $\frac{f(x) - g(x)}{f(x)}$ стремится к нулю при $x\to x_0$:

$$ \lim\limits_{x\to x_0} \frac{f(x) - g(x)}{f(x)} =0 .$$

В этом смысле функции, эквивалентные заданной, при- ближают ее лучше, чем другие функции.

Например, функции $x, \frac{1}{2}x, 2x, 10x$ являются бесконечно малыми при $ x \to 0$, так же как и sin $x$, а поэтому абсолютные погрешности при замене sin $x$ каждой из них стремятся к нулю при $ x \to 0$:
\[ \lim\limits_{x\to x_0} (\sin x - x) = \lim\limits_{x\to x_0} (\sin x -  \frac{1}{2} x) =\] 
\[ =\lim\limits_{x\to x_0} (\sin x - 2x) = \lim\limits_{x\to x_0} (\sin x - 10x) = 0.\]

Но лишь одна из всех перечисленных функций, а именно $g(x) = x$, обладает тем свойством, что относительная погреш-

\end{document}
